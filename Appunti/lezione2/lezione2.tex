\chapter{Lezione 2}
\section{La cinematica}
La cinematica è una parte della fisica che si occupa di descrivere il moto. In particolare, utilizzeremo l'algebra dei vettori vista nella lezione precedente e cercheremo, appunto, di descrivere il moto.

Durante il corso, ci limiteremo a studiare il caso in cui l'oggetto che si muove è un \textbf{corpo puntiforme}; questo vuol dire che le dimensioni di tale oggetto sono trascurabili, rispetto al moto che compie. Un'altra considerazione da fare è che faremo una trattazione, che vale nelle misure in cui le velocità che caratterizzano i moti che studieremo, sono molto più piccole della velocità della luce.

\subsection{Definizione di moto}
Il moto è un concetto relativo (nel senso che un corpo è in moto rispetto a un altro); \textbf{un corpo è in un moto relativo rispetto a un altro, quando cambia nel tempo la sua posizione rispetto all'altro corpo}. La condizione opposta al moto si chiama \textbf{quiete}. 

\subsection{Sistema di riferimento}
Prima di descrivere in termini matematici che cos'è il moto, dobbiamo introdurre il concetto di sistema di riferimento.

Possiamo pensare ad un sistema di riferimento, come un'insieme di oggetti che sono in quiete uno rispetto all'altro (la loro posizione relativa non cambia), che sono in quiete rispetto all'osservatore che vuole descrivere il moto. 
\newpage
Questi oggetti sono generalmente rappresentati da una terna di assi cartesiani.

\begin{figure}[h]
\begin{center}
\includegraphics[width = 0.5 \textwidth]{lezione2/images/sistema riferimento1.png}
\label{fig:riferimento1}
\caption{Rappresentazione grafica di un sistema di riferimento}
\end{center}
\end{figure}

Possiamo scrivere gli assi cartesiani, come se fossero tre versori: In particolare, indichiamo l'asse delle $ x $ con il versore $ \overrightarrow{i} $ , le $ y $ con il versore $ \overrightarrow{j} $ e l'asse $ z $ con il versore $ \overrightarrow{k} $ .
L'origine del sistema di riferimento viene indicata con $ O $

\begin{figure}[h]
\begin{center}
\includegraphics[width = 0.5 \textwidth]{lezione2/images/sistema riferimento2}
\label{fig:riferimento2}
\end{center}
\end{figure}
\subsection{Descrivere il moto}
Se prendiamo come sistema di riferimento quello rappresentato in figura, abbiamo che la posizione di un oggetto a un certo istante $ t $ , è determinata da un \textbf{vettore posizione} definito con le componenti cartesiane. 
\newpage
Questo vettore posizione, rappresenta (in questo caso), la posizione iniziale dell'oggetto di cui stiamo descrivendo il moto. 

\begin{figure}[h]
\begin{center}
\includegraphics[width = 0.5 \textwidth]{lezione2/images/sistema riferimento3}
\label{fig:riferimento3}
\end{center}
\end{figure}

Abbiamo detto che un oggetto è in moto se la sua posizione (dove per posizione si intende proprio il vettore posizione) cambia nel tempo. Ciò vuol dire che: se chiamiamo $ t + \Delta t $ un certo intervallo di tempo trascorso dall'istante iniziale $ t $ , il vettore posizione sarà differente.

\begin{figure}[h]
\begin{center}
\includegraphics[width = 0.5 \textwidth]{lezione2/images/sistema riferimento4}
\label{fig:riferimento4}
\end{center}
\end{figure}

Per indicare di quanto è variato il vettore posizione, si introduce il \textbf{vettore spostamento}, che non è nient'altro che la differenza di questi due vettori posizione. Se chiamiamo $ \Delta \overrightarrow{r} $ il vettore spostamento
$$ \Delta \overrightarrow{r} = \overrightarrow{r} ( t + \Delta t) - \overrightarrow{r} ( t ) $$

Per rappresentare graficamente questo vettore, dobbiamo utilizzare la regola del parallelogramma; prima di procedere, però ricordiamoci che questa differenza che rappresenta il vettore spostamento, può essere riscritta come
$$ \Delta \overrightarrow{r} = \overrightarrow{r} ( t + \Delta t) + ( - \overrightarrow{r} ( t ) ) $$
\newpage
In pratica stiamo utilizzando il vettore opposto a $ \overrightarrow{r} (t) $

\begin{figure}[h]
\begin{center}
\includegraphics[width = 0.5 \textwidth]{lezione2/images/sistema riferimento5}
\label{fig:riferimento5}
\end{center}
\end{figure}

Si può verificare che se facciamo la somma  $ \delta \overrightarrow{r} + \overrightarrow{r} ( t ) $ effettivamente otteniamo il vettore $ \overrightarrow{r} ( t + \delta t ) $ , cioè il vettore posizione all'istante $ t + \Delta t $ .

Per descrivere il moto dobbiamo introdurre due quantità: la \textit{velocità} e l'\textit{accelerazione}.

Definiamo come \textit{Velocità Media} il rapporto tra il vettore spostamento e il tempo che è intercorso 
tra l'istante $ t + \Delta t$ (Sappiamo che se l'oggetto si muove il vettore spostamento è diverso da zero) . 


$Velocità\   media \overrightarrow{v}_{media} =\frac{\Delta \overrightarrow{r} }{\Delta t} $ 


Per come è definita, abbiamo la differenza di due vettori $\Delta \overrightarrow{r} $
moltiplicata per $\frac{1}{\Delta t}$ che è uno scalare, quindi il risultato è sempre
un vettore. 
Le unità di misura della velocità media sono quelle del modulo dello spostamento
diviso quelle del tempo, dimensionalmente quindi $[V_{media}= \frac{L}{T} ]$ dove $[\Delta\overrightarrow{r}]= L$
$[\Delta t]= T$ (Le parentesi quadre indicano che ci riferiamo alle unità di misura e non al valore).

La velocità media ovviamente non ci "soddisfa" a pieno perché qualunque spostamento
originato e termin  ato nello stesso punto, qualunque fosse il percorso, avrebbe
la stessa velocità media a parità del tempo impiegato.

Introduciamo la \textit{velocità istantanea} 
\begin{equation} 
    \overrightarrow{V}_{istantanea}= 
    \lim_{\Delta t \to 0} \overrightarrow{V}_{media} = 
    \lim_{\Delta t \to 0} \frac { \overrightarrow{r}(t+\Delta t) * \overrightarrow{r}(t) }{\Delta t} = 
    \diffp{\overrightarrow{r}}{t}
\end{equation}

Anche la velocità istantanea come quella media è un vettore e dato che
abbiamo $\frac {t +\Delta t} {t}$ dove $\lim_{\Delta t \to 0}$  è un limite che tende a $0$,
può essere pensata come una derivata.
Ovviamente  le dimensioni di \textit{velocità media}
saranno sempre $[V]=\frac{L}{T}$.
 Essendo un vettore possiamo esprimere la velocità media
 secondo le sue componenti cartesiane 
 $\overrightarrow{v}= v_{x}\overrightarrow{i} + v_{y}\overrightarrow{j} + v_{z}\overrightarrow{k}$

Le componenti sono legate alla velocità in questo modo:
$v^2= \overrightarrow{v} \prodscal\overrightarrow{v}= v_{x}^2+v_{y}^2+v_{z}^2$

Per descrivere la posizione del corpo puntiforme, come
così come abbiamo introdotto prima la velocità media e poi la
velocità istantanea, analogamente anche il vettore velocità istantanea
cambia nel tempo. Abbiamo bisogno di qualcosa che ci dica \textit{di quanto}
cambia la velocità istantanea. 


Se in un istante $t$ ha velocità $\overrightarrow{v}(t)$, 
dopo un certo tempo avrà  $\overrightarrow{v}(t +\Delta t)$ e quindi 
possiamo introdurre l'\textit{accelerazione media}
\begin{equation}
    \overrightarrow{a}=
    \frac{\overrightarrow{v}(t +\Delta t) - \overrightarrow{v}(t)}{\Delta t} 
\end{equation}

Le unità di misura sono
 $[a_{media}]=\frac{[v]}{[\Delta t]}= \frac{L/T}{T}=\frac{L}{T^2}$

Anche in questo caso, come abbiamo fatto per la velocità,  anziché considerare l'accelerazione media possiamo vedere cosa capita, quando la differenza di tempi diventa sempre più piccola facendo di nuovo un limite:
$$ a_{istantanea} = \lim_{ \Delta t \to 0} {\overrightarrow{a}_{media} } = \overrightarrow{a}  $$
Chiameremo questa quantità \textit{accelerazione istantanea} o più semplicemente \textit{accelerazione}.
Anche l'accelerazione può essere descritta con le componenti cartesiane:
$$ \overrightarrow{a} = a_{x} \overrightarrow{i} + a_{y} \overrightarrow{j} + a_{z} \overrightarrow{k} $$
Inoltre il modulo quadro dell'accelerazione è definito come il prodotto scalare dell'accelerazione con se stessa
$$ a^{2} = \overrightarrow{a} \prodscal \overrightarrow{a} = {a^{2}}_{x} + {a^{2}}_{y} + {a^{2}}_{z} $$
Facendo poi la radice quadrata di questa quantità, ricaviamo il modulo dell'accelerazione.

Questo, è tutto ciò che ci serve per descrivere il moto; ricapitolando per descrivere il moto dobbiamo:
\begin{itemize}
	\item introdurre un sistema di riferimento
	\item descrivere il corpo che si muove all'interno di questo sistema con un vettore posizione
	\item se la posizione cambia nel tempo, dobbiamo introdurre un vettore spostamento
	\item lo spostamento per unità di tempo, perso per $\Delta t \to 0$ definisce la velocità istantanea (anch'essa è un vettore)
	\item se la velocità istantanea varia nel tempo,  prendendo il limite in cui $\Delta t \to 0$,  possiamo introdurre la variazione della velocità istantanea fatta rispetto al tempo, ottenendo un altro vettore chiamato accelerazione istantanea o più semplicemente accelerazione
\end{itemize}

Utilizzeremo adesso le quantità viste fin'ora per descrivere il \textbf{moto rettilineo} e il \textbf{moto circolare}.

\subsection{Moto rettilineo}
Il moto rettilineo è un moto in cui il nostro corpo puntiforme, 
si muove lungo una retta; tipicamente la retta che viene scelta, 
lungo la quale avviene il moto è l'asse delle $ x $.
Dobbiamo, quindi introdurre un sistema di riferimento 
(in questo caso scegliamo l'asse delle $ x $ , 
perché è la retta lungo la quale avviene il moto),
 e dobbiamo scegliere un'origine.
\newpage
Supponiamo che il nostro sistema di riferimento, 
sia quello rappresentato in figura, dove $ x = 0 $ 
rappresenta l'origine.

\begin{figure}[h]
\begin{center}
\includegraphics[width = 0.5 \textwidth]{lezione2/images/rettilineo1}
\label{fig:rettilineo1}
\end{center}
\end{figure}

Da prima sappiamo che il vettore posizione 
$\overrightarrow{r}(t)=x(t)\overrightarrow{i}+y(t)\overrightarrow{j}+z(t)\overrightarrow{k} $
e che la velocità è 
$\overrightarrow{v}(t)=v_{x}(t)\overrightarrow{i}+v_y(t)\overrightarrow{j}+v_z(t)\overrightarrow{k}$
e analogamente l'accelerazione
$\overrightarrow{a}(t)=a_{x}(t)\overrightarrow{i}+a_y(t)\overrightarrow{j}+a_z(t)\overrightarrow{k}$

Se il moto avviene solamente lungo l'asse delle $x$ allora la coordinata $y$
non cambierà mai e se non cambia allora la variazione della coordinata y sarà $0$
e la variazione della coordinata y mi darà la componente lungo 
y della velocità. 

Ma se la componente lungo y della velocità non cambia mai, allora anche la
componente lungo y dell'accelerazione sarà $0$.
Quindi nelle formule prcedenti avremo $v_y=0 \; e \;  v_z=0$
e anche nell'accelerazione $a_y=0 \; e \; a_z=0$.\\

Quindi per studiare il moto rettilineo l'unica parte interessante
è la componente delle $x$ ignorando le altre:

$\overrightarrow{r}(t)=x(t)\overrightarrow{i}$

$\overrightarrow{v}(t)=v_{x}(t)\overrightarrow{i}$ 

$\overrightarrow{a}(t)=a_{x}(t)\overrightarrow{i}$ 

ricordandoci sempre che sono dei vettori, 
in cui $v_x$ ci dice \textit{quanto} e in che direzione
(positiva o negativa) spostiamo 
il versore $\overrightarrow{i}$.

Per la velocità sappiamo che è la derivata della posizione rispetto
al tempo:

\begin{equation}
\overrightarrow{v}=\diffp{\overrightarrow{r}}{t} \rightarrow v_x=\diffp{x}{t}
\end{equation}

e l'accelerazione sarà:

\begin{equation}
    \overrightarrow{a}=\diffp{\overrightarrow{v}}{t} \rightarrow a_x=\diffp{V_x}{t}
\end{equation}

Possiamo notare come non ci sia più la componente y e z per cui:

$v=\diffp{x}{t}$

$a=\diffp{v}{t}=\diffp{dx}{dt}=\frac{d^2x}{dt^2}$

Quindi osserviamo che x è una variabile in funzione del tempo, 
v la derivata prima di x rispettto al tempo e a la derivata seconda.

Riassumendo le equazioni del moto rettilineo sono:

$x(t)$

$v=\diffp{x}{t}$

$a=\frac{dv}{dt}=\frac{d^2x}{d t^2}$

Di solito i problemi nel moto rettilineo si dividono in 

\begin{itemize}
\item conosciuamo la posizone
$x(t)$ (detta legge oraria) e vogliamo trovare la velocità e l'accelerazione.

\item Conosciamo l'accellerazione e vogliamo velocità e legge oraria
\end{itemize}

Nella seconda tipologia dovremo ricavare la velocità usando l'integrale
(che è l'inverso delle derivata) , passando quindi dalla derivata seconda(accellerazione)
a derivata prima (velocità) di $x(t)$.

(Se a una funzione x applichiamo l'integrale ottenendo
una fuznione y 
allora y è una \textit{primitiva} di x )

Successivamente integrando di nuovo passiamo dalla velocità alla 
legge oraria $x(t)$.

% gather* permette di mettere più equazioni nello stesso ambiente senza numerarle
% gather invece le numera
\begin{equation}
    a(t)=\frac{dv}{dt} \rightarrow v(t)= \int {a(t) dt} +c1 \rightarrow v(t=t_0)
\end{equation}

\begin{equation}
    v(t)=\frac{dx}{dt} \rightarrow x(t)= \int {v(t)dt} +c2 \rightarrow x(t=t_0)
\end{equation}


 Risolvere l'integrale può essere complicato o facile a seconda della funzione
 da integrare.
Il motivo per cui aggiungiamo le costanti c1 e c2 è
che nel caso generale vanno messe ma poi quando deriviamo 
essendo costanti valgono $0$.

Dal punto di vista fisica significa che l'integrale è una
funzione del tempo, mentre la costante c inica il punto da cui siamo
partiti

\begin{equation}
   v(t)= \int {a(t)dt} +c1 \rightarrow 
   \frac{d}{dt}[\int{a(t)dt}+c1]=
   \frac{d}{dt}\int{a(t)dt}\rightarrow \frac{d}{dt}c1=a(t)
\end{equation}

Riassumendo:

Condizioni iniziali $t=0$
$v(t=0)=v_0$ e $x(t_0)=x_0$

\begin{center}
    \begin{tabular}{ |c|c|c|c| } 
     \hline
     $a(t)$ & $v(t)$ & $v(t)$ & Tipo di moto \\ 
     \hline
        $a\neq 0 (costante)$ & $at+v_0$ & $\frac{1}{2}at^2+v_0t+x_0$ & Moto rettilineo unif. acc. \\ 
     $0$ & $v_0 \neq 0$ & $v_0t+x_0$ & Moto ret. uniforme \\ 
     $0$ & $0$ & $0$ & quiete \\
     \hline
    \end{tabular}
\end{center}
