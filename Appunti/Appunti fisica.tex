\documentclass[a4paper]{book}
\usepackage[T1]{fontenc}
\usepackage[utf8]{inputenc} 
\usepackage[italian]{babel}
\usepackage{eucal,enumitem}
\usepackage{graphicx}
\usepackage{caption}
\usepackage{refstyle}
\usepackage{amsmath,amssymb,amsthm}
\usepackage{tikz}
\usepackage{cancel} 
\usepackage{multicol}
\usepackage{color}
\usepackage{fancyhdr}
\usepackage{mathtools}
\usepackage[thinc]{esdiff}
\usepackage{physics}
\newcommand{\prodscal}{\text{\large{\textbf{\textperiodcentered}}}} %trovato su internet per definire simbolo prodotto scalare. Da usare con \prodscal

\theoremstyle{definition}
\newtheorem{definizione}{Definizione}[chapter]
\theoremstyle{remark}
\newtheorem{esempio}{\textbf{Esempio}}
\theoremstyle{plain}
\newtheorem{proposizione}{Proposizione}[chapter]

\begin{document}
\title{Appunti Fisica}
\author{Valerio Ghirardotto}
\maketitle
\tableofcontents
\include{lezione1/lezione1}

\chapter{Lezione 2}
\section{La cinematica}
La cinematica è una parte della fisica che si occupa di descrivere il moto. In particolare, utilizzeremo l'algebra dei vettori vista nella lezione precedente e cercheremo, appunto, di descrivere il moto.

Durante il corso, ci limiteremo a studiare il caso in cui l'oggetto che si muove è un \textbf{corpo puntiforme}; questo vuol dire che le dimensioni di tale oggetto sono trascurabili, rispetto al moto che compie. Un'altra considerazione da fare è che faremo una trattazione, che vale nelle misure in cui le velocità che caratterizzano i moti che studieremo, sono molto più piccole della velocità della luce.

\subsection{Definizione di moto}
Il moto è un concetto relativo (nel senso che un corpo è in moto rispetto a un altro); \textbf{un corpo è in un moto relativo rispetto a un altro, quando cambia nel tempo la sua posizione rispetto all'altro corpo}. La condizione opposta al moto si chiama \textbf{quiete}. 

\subsection{Sistema di riferimento}
Prima di descrivere in termini matematici che cos'è il moto, dobbiamo introdurre il concetto di sistema di riferimento.

Possiamo pensare ad un sistema di riferimento, come un'insieme di oggetti che sono in quiete uno rispetto all'altro (la loro posizione relativa non cambia), che sono in quiete rispetto all'osservatore che vuole descrivere il moto. 
\newpage
Questi oggetti sono generalmente rappresentati da una terna di assi cartesiani.

\begin{figure}[h]
\begin{center}
\includegraphics[width = 0.5 \textwidth]{lezione2/images/sistema riferimento1.png}
\label{fig:riferimento1}
\caption{Rappresentazione grafica di un sistema di riferimento}
\end{center}
\end{figure}

Possiamo scrivere gli assi cartesiani, come se fossero tre versori: In particolare, indichiamo l'asse delle $ x $ con il versore $ \overrightarrow{i} $ , le $ y $ con il versore $ \overrightarrow{j} $ e l'asse $ z $ con il versore $ \overrightarrow{k} $ .
L'origine del sistema di riferimento viene indicata con $ O $

\begin{figure}[h]
\begin{center}
\includegraphics[width = 0.5 \textwidth]{lezione2/images/sistema riferimento2}
\label{fig:riferimento2}
\end{center}
\end{figure}
\subsection{Descrivere il moto}
Se prendiamo come sistema di riferimento quello rappresentato in figura, abbiamo che la posizione di un oggetto a un certo istante $ t $ , è determinata da un \textbf{vettore posizione} definito con le componenti cartesiane. 
\newpage
Questo vettore posizione, rappresenta (in questo caso), la posizione iniziale dell'oggetto di cui stiamo descrivendo il moto. 

\begin{figure}[h]
\begin{center}
\includegraphics[width = 0.5 \textwidth]{lezione2/images/sistema riferimento3}
\label{fig:riferimento3}
\end{center}
\end{figure}

Abbiamo detto che un oggetto è in moto se la sua posizione (dove per posizione si intende proprio il vettore posizione) cambia nel tempo. Ciò vuol dire che: se chiamiamo $ t + \Delta t $ un certo intervallo di tempo trascorso dall'istante iniziale $ t $ , il vettore posizione sarà differente.

\begin{figure}[h]
\begin{center}
\includegraphics[width = 0.5 \textwidth]{lezione2/images/sistema riferimento4}
\label{fig:riferimento4}
\end{center}
\end{figure}

Per indicare di quanto è variato il vettore posizione, si introduce il \textbf{vettore spostamento}, che non è nient'altro che la differenza di questi due vettori posizione. Se chiamiamo $ \Delta \overrightarrow{r} $ il vettore spostamento
$$ \Delta \overrightarrow{r} = \overrightarrow{r} ( t + \Delta t) - \overrightarrow{r} ( t ) $$

Per rappresentare graficamente questo vettore, dobbiamo utilizzare la regola del parallelogramma; prima di procedere, però ricordiamoci che questa differenza che rappresenta il vettore spostamento, può essere riscritta come
$$ \Delta \overrightarrow{r} = \overrightarrow{r} ( t + \Delta t) + ( - \overrightarrow{r} ( t ) ) $$
\newpage
In pratica stiamo utilizzando il vettore opposto a $ \overrightarrow{r} (t) $

\begin{figure}[h]
\begin{center}
\includegraphics[width = 0.5 \textwidth]{lezione2/images/sistema riferimento5}
\label{fig:riferimento5}
\end{center}
\end{figure}

Si può verificare che se facciamo la somma  $ \delta \overrightarrow{r} + \overrightarrow{r} ( t ) $ effettivamente otteniamo il vettore $ \overrightarrow{r} ( t + \delta t ) $ , cioè il vettore posizione all'istante $ t + \Delta t $ .

Per descrivere il moto dobbiamo introdurre due quantità: la \textit{velocità} e l'\textit{accelerazione}.

Definiamo come \textit{Velocità Media} il rapporto tra il vettore spostamento e il tempo che è intercorso 
tra l'istante $ t + \Delta t$ (Sappiamo che se l'oggetto si muove il vettore spostamento è diverso da zero) . 


$Velocità\   media \overrightarrow{v}_{media} =\frac{\Delta \overrightarrow{r} }{\Delta t} $ 


Per come è definita, abbiamo la differenza di due vettori $\Delta \overrightarrow{r} $
moltiplicata per $\frac{1}{\Delta t}$ che è uno scalare, quindi il risultato è sempre
un vettore. 
Le unità di misura della velocità media sono quelle del modulo dello spostamento
diviso quelle del tempo, dimensionalmente quindi $[V_{media}= \frac{L}{T} ]$ dove $[\Delta\overrightarrow{r}]= L$
$[\Delta t]= T$ (Le parentesi quadre indicano che ci riferiamo alle unità di misura e non al valore).

La velocità media ovviamente non ci "soddisfa" a pieno perché qualunque spostamento
originato e termin  ato nello stesso punto, qualunque fosse il percorso, avrebbe
la stessa velocità media a parità del tempo impiegato.

Introduciamo la \textit{velocità istantanea} 
\begin{equation} 
    \overrightarrow{V}_{istantanea}= 
    \lim_{\Delta t \to 0} \overrightarrow{V}_{media} = 
    \lim_{\Delta t \to 0} \frac { \overrightarrow{r}(t+\Delta t) * \overrightarrow{r}(t) }{\Delta t} = 
    \diffp{\overrightarrow{r}}{t}
\end{equation}

Anche la velocità istantanea come quella media è un vettore e dato che
abbiamo $\frac {t +\Delta t} {t}$ dove $\lim_{\Delta t \to 0}$  è un limite che tende a $0$,
può essere pensata come una derivata.
Ovviamente  le dimensioni di \textit{velocità media}
saranno sempre $[V]=\frac{L}{T}$.
 Essendo un vettore possiamo esprimere la velocità media
 secondo le sue componenti cartesiane 
 $\overrightarrow{v}= v_{x}\overrightarrow{i} + v_{y}\overrightarrow{j} + v_{z}\overrightarrow{k}$

Le componenti sono legate alla velocità in questo modo:
$v^2= \overrightarrow{v} \prodscal\overrightarrow{v}= v_{x}^2+v_{y}^2+v_{z}^2$

Per descrivere la posizione del corpo puntiforme, come
così come abbiamo introdotto prima la velocità media e poi la
velocità istantanea, analogamente anche il vettore velocità istantanea
cambia nel tempo. Abbiamo bisogno di qualcosa che ci dica \textit{di quanto}
cambia la velocità istantanea. 


Se in un istante $t$ ha velocità $\overrightarrow{v}(t)$, 
dopo un certo tempo avrà  $\overrightarrow{v}(t +\Delta t)$ e quindi 
possiamo introdurre l'\textit{accelerazione media}
\begin{equation}
    \overrightarrow{a}=
    \frac{\overrightarrow{v}(t +\Delta t) - \overrightarrow{v}(t)}{\Delta t} 
\end{equation}

Le unità di misura sono
 $[a_{media}]=\frac{[v]}{[\Delta t]}= \frac{L/T}{T}=\frac{L}{T^2}$

Anche in questo caso, come abbiamo fatto per la velocità,  anziché considerare l'accelerazione media possiamo vedere cosa capita, quando la differenza di tempi diventa sempre più piccola facendo di nuovo un limite:
$$ a_{istantanea} = \lim_{ \Delta t \to 0} {\overrightarrow{a}_{media} } = \overrightarrow{a}  $$
Chiameremo questa quantità \textit{accelerazione istantanea} o più semplicemente \textit{accelerazione}.
Anche l'accelerazione può essere descritta con le componenti cartesiane:
$$ \overrightarrow{a} = a_{x} \overrightarrow{i} + a_{y} \overrightarrow{j} + a_{z} \overrightarrow{k} $$
Inoltre il modulo quadro dell'accelerazione è definito come il prodotto scalare dell'accelerazione con se stessa
$$ a^{2} = \overrightarrow{a} \prodscal \overrightarrow{a} = {a^{2}}_{x} + {a^{2}}_{y} + {a^{2}}_{z} $$
Facendo poi la radice quadrata di questa quantità, ricaviamo il modulo dell'accelerazione.

Questo, è tutto ciò che ci serve per descrivere il moto; ricapitolando per descrivere il moto dobbiamo:
\begin{itemize}
	\item introdurre un sistema di riferimento
	\item descrivere il corpo che si muove all'interno di questo sistema con un vettore posizione
	\item se la posizione cambia nel tempo, dobbiamo introdurre un vettore spostamento
	\item lo spostamento per unità di tempo, perso per $\Delta t \to 0$ definisce la velocità istantanea (anch'essa è un vettore)
	\item se la velocità istantanea varia nel tempo,  prendendo il limite in cui $\Delta t \to 0$,  possiamo introdurre la variazione della velocità istantanea fatta rispetto al tempo, ottenendo un altro vettore chiamato accelerazione istantanea o più semplicemente accelerazione
\end{itemize}

Utilizzeremo adesso le quantità viste fin'ora per descrivere il \textbf{moto rettilineo} e il \textbf{moto circolare}.

\subsection{Moto rettilineo}
Il moto rettilineo è un moto in cui il nostro corpo puntiforme, si muove lungo una retta; tipicamente la retta che viene scelta, lungo la quale avviene il moto è l'asse delle $ x $.
Dobbiamo, quindi introdurre un sistema di riferimento (in questo caso scegliamo l'asse delle $ x $ , perché è la retta lungo la quale avviene il moto), e dobbiamo scegliere un'origine.
\newpage
Supponiamo che il nostro sistema di riferimento, sia quello rappresentato in figura, dove $ x = 0 $ rappresenta l'origine.

\begin{figure}[h]
\begin{center}
\includegraphics[width = 0.5 \textwidth]{lezione2/images/rettilineo1}
\label{fig:rettilineo1}
\end{center}
\end{figure}






\chapter{Lezione 3} % leggi di newton
\chapter{Lezione 4} % Elettrostatica: carica elettrica e legge coulomb
\chapter{Lezione 5} % Campo elettrico
\chapter{Lezione 6} % Legge di Gauss
\chapter{Lezione 7}
\section{Lavoro ed energia cinetica.}
\subsection{Lavoro}
Abbiamo una nozione intuitiva di lavoro che si collega alla nostra idea di fatica e/o dispendio energetico. Un esempio adatto che ci collega subito al concetto di lavoro è quello dello spostamento di un oggetto da un punto A ad un punto B. Per farlo impieghiamo energia in misura di quanto è grande la distanza tra i punti di partenza e arrivo. Tanta distanza = tanta fatica. Il dispendio energetico avviene solo relativamente allo spostamento, cioè non dobbiamo più spendere energia una volta raggiunto il punto finale. Per esempio se solleviamo con una carrucola un oggetto fino ad una certa altezza, possiamo fissarla in quel punto senza più spendere energia. Non conta in questo spostamento solo la forza e la distanza però, ma anche la componente della forza applicata nella direzione dello spostamento. 

Da queste considerazioni possiamo introdurre il concetto di lavoro.
\begin{figure}[h!]
	\begin{center}
		\includegraphics[width=6cm]{lezione7/images/1Precorsolavoroedenergia}\\
		\caption{Lavoro per spostare un oggetto applicando una forza costante in modulo, direzione e verso}
	\end{center}
\end{figure}
\begin{definizione}
	
	Il lavoro è dato da una forza applicata costante in modulo e direzione da 
	$$L=\vec{F}\cdot \vec{\Delta s}=F\cdot \Delta s \cdot cos \theta$$
	
Se invece la forza applicata cambia lungo una curva allora punto per punto avremo la situazione appena descritta, e passando al continuo otterremo un integrale. Nello specifico la situazione è data dalla figura 
	\begin{figure}[h!]
	\begin{center}
		\includegraphics[width=6cm]{lezione7/images/2Precorsolavoroedenergia}\\
		\caption{Lavoro per spostare un oggetto applicando una forza costante in modulo ma non in direzione e verso, decomposizione infinitesima}
	\end{center}
\end{figure}

ed è necessario calcolare un integrale (perchè stiamo lavorando una suddivisione in tratti infinitesimi) per ottenere il lavoro lunga la curva da A a B dato da 
$$L_{AB}=\int_{A}^{B} \vec{F} \cdot d \vec{s}$$ 

Il lavoro avrà come unità di misura il Joule J, ottenuto così:
$$[L]=[F][\Delta s] = \frac{ML}{T^2} L= \frac{ML^2}{T^2} = \frac{kg \cdot m^2}{s^2}:=J$$

dove abbiamo indicato con $L$ la lunghezza, $M$ la massa e $T$ il tempo. E' importante ricordarsi che il lavoro è una quantita \textbf{scalare}.
\end{definizione}

 Possiamo analizzare i seguenti casi particolari facendo variare la direzione e verso della forza 
\begin{figure}[h]
	\begin{center}
		\includegraphics[width=10cm]{lezione7/images/3Precorsolavoroedenergia}\\
		\caption{Lavoro al variare dell'angolo $\theta$.}
	\end{center}
\end{figure}

Quindi parliamo di lavoro positivo, nullo e negativo rispettivamente.

\subsection{Energia cinetica}
Abbiamo introdotto il lavoro a partire dal fatto che su un oggetto spostato agisce una forza. Ma per la seconda legge di Newton, se su un corpo agisce una forza la velocità di questo oggetto varia secondo la legge $F =m\cdot a$. Allora possiamo pensare di introdurre una grandezza, in funzione della velocità, per fare in modo tale che variando la forza, anche la velocità varia di conseguenza, e quindi la grandezza definita, e cambia perché stiamo compiendo un lavoro.


\begin{esempio}
	Consideriamo un caso particolare per introdurre questa grandezza. L'esempio che consideriamo è quello del moto rettilineo uniformemente accellerato, cioè soggetto ad una forza applicata costante in modulo direzione e verso. In particolare la direzione è proprio quella del moto.
	
	\begin{figure}[h]
		\begin{center}
			\includegraphics[width=8cm]{lezione7/images/4Precorsolavoroedenergia.jpg}\\
			\caption{Moto rettilineo uniformemente accelerato}
		\end{center}
		\end{figure}
	
	Bisogna ricordare che nel caso di moto rettilineo uniformemente accelerato valgono le relazioni
	
	$$x(t) =\frac{1}{2}at^2+v_0 t +x_0$$
	$$v(t)=at+v_0$$.
	
	Perciò se $a=F_0 /m$, $x_0 =x_A$ e $v_0 =v_A$ 
		\begin{figure}[h]
		\begin{center}
			\includegraphics[width=11cm]{lezione7/images/5Precorsolavoroedenergia}\\
			\caption{Conti}
		\end{center}
	\end{figure}
\end{esempio}

\begin{definizione}
	L'energia cinetica viene definita come 
	$$E_k=\frac{1}{2}mv^2$$
	
	La sua unità di misura è ancora il Joule e si ha che 
	
	$$L_{AB}=E_k (B)- E_k (A)$$ 
	e quest'ultima è una relazione generale.
\end{definizione}

\subsection{Forze conservative ed energia potenziale}

\begin{definizione}
Una forza si dice conservativa se il lavoro svolto non dipende da una particolare curva ma solamente dai punti iniziali e finali. 
\end{definizione}
	\begin{figure}[h]
	\begin{center}
		\includegraphics[width=8cm]{lezione7/images/6Precorsolavoroedenergia}\\
		\caption{Cammini chiusi.}
	\end{center}
\end{figure}

\begin{esempio}[La forza peso è conservativa]
	\begin{figure}[h]
	\begin{center}
		\includegraphics[width=10cm]{lezione7/images/7Precorsolavoroedenergia}\\
		\caption{Lavoro della forza peso}
	\end{center}
\end{figure}

\end{esempio}

Se una forza è conservativa, perciò dipende solo dalla posizione iniziale e finale e possiamo introdurre una funzione della sola coordinata spaziale, la posizione, detta \textit{Energia potenziale}. Cioè stiamo dicendo che nel caso di una forza conservativa 
$$L_{AB}=\int_{A}^{B} \vec{F}\cdot d \vec{s} = E_p (A)- E_p (B)$$

Dove con il simbolo $E_p$ indichiamo l'energia potenziale in un certo punto (quindi funzione solo della posizione). 

\begin{itemize}
\item Questa funzione è definita solo a meno di una costante additiva, in quanto definita tramite una differenza, quindi la costante non influisce in alcun modo. Infatti se sostituisco $E_p (\vec{r})$ con $E_p (\vec{r})+E_{p,0} $ allora $E' _p (A)=E_p (A)+E_{p,0} $, $E' _p (B)=E_p (B)+E_{p,0} $ e resta definita la differenza in quanto 
$$E' _p (A) - E' _p (B)= E' _p (A)=E_p (A)+E_{p,0} -(E_p (B)+E_{p,0}) = E _p (A) - E _p (B)$$

\item Abbiamo espresso il lavoro come variazione di energia potenziale dal punto iniziale al punto finale e non viceversa, perciò stiamo 'scegliendo' un segno. Lo facciamo in questo modo perché il lavoro che una forza $F$ compie per spostare  un oggetto da A a B, è uguale ed opposto a quello di una agente esterno che vuole portarlo da B ad A. Allora questa inversione inverte gli estremi dell'integrale che definisce il lavoro, allora inverte un segno. Allora se la forza è conservativa il lavoro che l'esterno fa contro la forza $F$ per uno spostamento da B ad A se è positivo l'energia potenziale dell'oggetto spostato è aumentata (che è quello che penseremo intuitivamente).
\item Nel caso di forze conservative il lavoro su spostamenti nulli è ovviamente nullo. Cioè se punto iniziale e finale sono uguali, il lavoro è nullo.
\end{itemize}

Quindi in generale abbiamo che il lavoro può essere espresso tramite 
$$L_{AB}=E_k (B)- E_k (A)$$ 

Nel caso particolare di forze conservative abbiamo 
$$L_{AB}= E_p (B)- E_p (A)$$

che deve dunque essere uguale alla prima espressione richiamata

$$E_k (B)- E_k (A)= E_p (B)- E_p (A)$$ 

da cui 


\begin{proposizione}[Conservazione dell'energia]
	L'energia totale/meccanica per forze conservative si conserva, cioè vale che 
	
	$$E_k (B)+E_p (B)= E_k (A)+E_p (A)$$ 
\end{proposizione}

\begin{esempio}
	Un carrello scivola avanti e indietro lungo un binario liscio posto in un piano verticale. la forma del binario è parabolica e la quota in funzione della positione orizzontale è $y=kx^2$ con $k=0,92$.
	\begin{enumerate}
		\item Trovare le coordinate $x$ dei punti di inversione del moto se la velocità massima del carrello vale $v_{max}=8,5 m/s$.
		\item Trovare il modulo della velocità del carrello quando si trova a metà dell'altezza massima.
	\end{enumerate}
	\begin{figure}[h]
	\begin{center}
		\includegraphics[width=12cm]{lezione7/images/8Precorsolavoroedenergia.jpg}\\
		\caption{Rappresentazione del problema.}
	\end{center}
\end{figure}
	\begin{figure}[h]
	\begin{center}
		\includegraphics[width=12cm]{lezione7/images/9Precorsolavoroedenergia.jpg}\\
		\caption{}
	\end{center}
\end{figure}

	\begin{figure}[h]
	\begin{center}
		\includegraphics[width=12cm]{lezione7/images/10Precorsolavoroedenergia.jpg}\\
		\caption{Svolgimento punto 1.}
	\end{center}
\end{figure}

	\begin{figure}[h]
	\begin{center}
		\includegraphics[width=12cm]{lezione7/images/11Precorsolavoroedenergia.jpg}\\
		\caption{Svolgimento punto 2}
	\end{center}
\end{figure}

\end{esempio}

\end{document}
